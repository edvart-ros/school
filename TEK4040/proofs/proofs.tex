\documentclass{article}
\usepackage[english]{babel}
\usepackage{amsmath}
\usepackage{amssymb}
\usepackage{mathtools}
\usepackage{amsthm}
\usepackage{cancel}
\usepackage{soul}


\newtheorem{theorem}{Theorem}
\newtheorem{prop}{Problem}

\begin{document}

\newcommand{\q}{\vec}
\newcommand{\aq}{\underline}
\newcommand{\qs}[1]{\vec{#1}^*}
\newcommand{\nsum}[1]{\sum_{#1=1}^n}
\newcommand\algeq{\stackrel{\mathclap{\normalfont\mbox{alg.}}}{\iff}}



\begin{theorem}
    Given a basis $\{\q{p}_i\}$ with dual basis $\{\qs{p}_i\}$, any vector $\q{r}$ can be written as:
    $$\vec{r} = \sum_{i=1}^n r_i^p\q{p}_i, \quad r_i^p = \left<\q{r}, \qs{p}_i\right>$$
\end{theorem}
\begin{proof}
    Since $\{\q{p}_i\}$ forms a basis, we can write $\q{r}$ as:
    $$\q{r} = \sum_{i=1}^nr_i^p\q{p}_i$$
    Now, consider the inner product of an arbitrary vector $\q{r}$ and $\qs{p}_i$:
    \begin{align*}
        \left<\q{r}, \qs{p}_i\right> &= \left<\sum_{j=1}^nr_j^p\q{p}_j, \qs{p}_i\right>  \\
        &= \sum_{j=1}^nr_j^p\left<\q{p}_j, \qs{p}_i \right> \\
        &= \nsum{j}r_j^p \delta_{ij} = r_i^p
    \end{align*}
    Thus:
    \begin{equation*}
        r_i^p = \left<\q{r}, \qs{p}_i\right>
    \end{equation*}
\end{proof}

\begin{theorem}
Given the basis $\{\q{p}_i\}$ for the vector space $\mathcal{V}$, any linear operator
$\mathbb{A}$ can be represented in $\mathbb{R}^{n\times n}$ as the matrix $A^p$:
$$[\mathbb{A}]^p = A^p = \left[\left<\mathbb{A}\q{p}_j, \qs{p_i}\right>\right]$$
$$\q{y} = \mathbb{A}\q{x} \algeq \aq{y}^p = A^p\aq{x}^p$$
\end{theorem}

\begin{proof}
    Let $\mathbb{A}$ be a linear operator on $\mathcal{V}$, with the basis $\{\q{p}_i\}$ and dual basis
    $\{\qs{p}_i\}$. Then, let $\q{y} = \mathbb{A}\q{x}$
    Again, we express $\q{y}$ and $\q{x}$ in the p-frame:
    $$\q{y} = \nsum{i}y_i^p\q{p}_i, \quad \q{x} = \nsum{i}x_i^p\q{p}_i$$ 
    From theorem 1, we have:
    \begin{align*}
        y_i^p &= \left<\q{y}, \qs{p}_i\right> \\
        &= \left<\mathbb{A}\q{x}, \qs{p}_i\right> \\
    \end{align*}
    Expressing $\q{x}$ in the p-frame and using the linearity of $\mathbb{A}$, we get
    \begin{align*}
        &= \left< \mathbb{A}\nsum{j}x_j^p\q{p}_j, \qs{p}_i\right> \\
        &= \left< \nsum{j}x_j^p\mathbb{A}\q{p}_j, \qs{p}_i\right> \\
        y_i^p&= \nsum{j}x_j^p\left<\mathbb{A}\q{p}_j, \qs{p}_i\right>
    \end{align*}
    This can be written as the following matrix equation
    $$\begin{bmatrix}
        y_1 \\ \vdots \\ y_n
    \end{bmatrix} = 
    \begin{bmatrix}
        \left< \mathbb{A}\q{p}_1, \qs{p}_1\right>&  \hdots  & \left< \mathbb{A}\q{p}_n, \qs{p}_1\right> \\
        \vdots  & \ddots & \vdots \\
        \left< \mathbb{A}\q{p}_n, \qs{p}_1\right> & \hdots & \left< \mathbb{A}\q{p}_n, \qs{p}_n\right>
    \end{bmatrix}
    \begin{bmatrix}
        x_1^p \\ \vdots \\ x_n^p
    \end{bmatrix}$$
    or:
    $$\aq{y}^p = A^p\aq{x}^p$$ 
    where
    $$ A^p = \left[\left<\mathbb{A}\q{p}_j, \qs{p_i}\right>\right]$$
\end{proof}


\begin{prop}
    Find the matrix representation of the $" \q{\omega} \times "$ operator in the orthonormal frame $\{\q{p}_i\}$
\end{prop}
\subsubsection*{Sol.}
To find the matrix representation we use Theorem 2:
\begin{align*}
    A^p &= \left[\left<\mathbb{A}\q{p}_j, \qs{p_i}\right>\right] \\
    &=  \left[\left<\q{\omega} \times \q{p}_j, \qs{p_i}\right>\right]
\end{align*}
First we consider $\omega \times \q{p}_j$ by expressing $\omega$ in the p-frame:
\begin{align*}
    \q{\omega} \times \q{p}_j &= \left(\nsum{i}\omega_i^p\q{p}_i\right) \times \qs{p}_j  \\
    &= (\omega_1^p\q{p}_1 + \omega_2^p\q{p}_2 + \omega_3^p + \q{p}_3) \times \qs{p}_j \\
\end{align*}
Since the p-frame is orthonormal, $\qs{p}_i = \q{p}_i$:
\begin{align*}
    \q{\omega} \times \q{p_j} = (\omega_1^p\q{p}_1 + \omega_2^p\q{p}_2 + \omega_3^p\q{p}_3) \times \q{p}_j 
\end{align*}
This cross-product can be computed directly for $j = 1,2,3$:
\begin{align*}
    &\q{\omega}\times\q{p}_1 = -\omega_2^p\q{p}_3  + \omega_3^p\q{p}_2 \\
    &\q{\omega}\times\q{p}_2 = \omega_1^p\q{p}_3  - \omega_3^p\q{p}_1 \\
    &\q{\omega}\times\q{p}_3 = -\omega_1^p\q{p}_2  + \omega_2^p\q{p}_1 
\end{align*}

Then, computing the entries of the matrix $A^p$:

\begin{align*}
    &a_{ij} = \left<\q{\omega}\times\q{p}_j, \q{p}_i\right>\\
    &A^p = \begin{bmatrix}
        0 & -\omega_3^p & \omega_2^p \\
        \omega_3^p & 0 & -\omega_1^p \\
        -\omega_2^p &\omega_1^p& 0
    \end{bmatrix}
\end{align*}
This matrix is skew-symmetric and is defined by the coordinates of $\aq{\omega}^p$, so we denote the matrix representation of "$\q{\omega} \times$" as
$$\left[\q{\omega} \times \right]^p = S(\aq{\omega}^p)$$
$$\q{y} = \q{\omega}\times\q{x} \algeq \aq{y}^p = S(\aq{\omega}^p)\aq{x}^p$$


\begin{theorem}
    Given two bases $\{\q{p}_i\}$ and $\{\q{q}_i\}$ in $\mathcal{V}$. Let $\q{r}$ and $\mathbb{A}$ be a vector and a linear operator in $\mathcal{V}$, respectively.
    We then have the following relations between matrix representations in the two bases/frames:
    $$\aq{r}^q = C_p^q\aq{r}^p \quad \text{where} \quad C_p^q = \left[\left< \q{p}_j, \qs{q}_i\right>\right]$$
    $$\aq{r}^p = C_q^p\aq{r}^q \quad \text{where} \quad C_q^p = \left[\left< \q{q}_j, \qs{p}_i\right>\right]$$
    \vspace{0cm}
    $$ A^q = C_p^qA^pC_q^p \quad \text{and} \quad A^p = C_q^pA^qC_p^q$$
\end{theorem}
\begin{proof}
    Let $\aq{r}^q = C_p^q\aq{r}^p$ where $\aq{r}^q = \begin{bmatrix}
        r_1^q & \hdots &r_n^q
    \end{bmatrix}^T$ and $\aq{r}^p = \begin{bmatrix}
        r_1^p & \hdots \ r_n^p
    \end{bmatrix}^T$. Then,
    \begin{align*}
        r_i^q = \left<\q{r}, \qs{q}_i\right> = \nsum{j}C_{ij}r_j^p \\
    \end{align*}
    Rearranging and expressing $\q{r}$ in the p-frame:
    \begin{align*}
        \nsum{j}C_{ij}r_j^p = \left< \nsum{j}r_j\q{p}_j, \qs{q}_i \right>\\
    \end{align*}
    By linearity of the inner product, we get
    \begin{align*}
        \nsum{j}&C_{ij}r_j^p = \nsum{j}\left<\q{p}_j, \qs{q}_i\right>r_j^p \\
        \implies &C_{ij} = \left<\q{p}_j, \qs{q}_i\right> \\
    \end{align*}
    thus \begin{align*}
        \quad C_p^q &= \left[\left<\q{p}_j, \qs{q}_i\right>\right]
    \end{align*}
    The same procedure can be used to find $C_q^p$.

    Now, consider the equation $\q{y} = \mathbb{A}\q{x}$ where $\mathbb{A}$ is some linear operator on $\mathcal{V}$. We can coordinatize this equation in the p and q-frames:
    $$\aq{y}^q = A^q\aq{x}^q \quad \text{and} \quad \aq{y}^p = A^p\aq{x}^p$$
    From above, we also have the following relations
    $$\aq{y}^q = C_p^q\aq{y}^p \quad \text{and} \quad \aq{y}^p = C_q^p\aq{y}^q$$
    By substitution, we derive:
    \begin{align*}
        A^q\aq{x}^q &= \aq{y}^q \\
        &= C_q^p\aq{y}^p \\
        &= C_q^pA^p\aq{x}^p \\
        &= C_q^pA^pC_q^p\aq{x}^q
    \end{align*}
    And thus
    $$A^q = C_q^pA^pC_q^p$$
    Again, the same procedure can be used to show that $A^p = C_p^qA^qC_p^q$
\end{proof}

\begin{theorem}
    The Direction Cosine Matrix (DCM) between two frames whose bases are orthonormal, $R_p^q$, is an orthonormal matrix:
    $$(R_p^q)^{-1} = (R_p^q)^T$$
\end{theorem}

\begin{proof}
    From the definition of the DCM:
    \begin{equation*}
        R_p^q = \begin{bmatrix}{\aq{p}_1}^q & \aq{p}_2^q & \aq{p}_3^q\end{bmatrix}
    \end{equation*}
    \begin{equation*}
        (R_p^q)^T = \begin{bmatrix}
            (\aq{p}_1^q)^T\\
            (\aq{p}_2^q)^T\\
            (\aq{p}_3^q)^T\\
        \end{bmatrix}
    \end{equation*}
    Then, we compute $(R_p^q)^TR_p^q$
    \begin{align*}
        (R_p^q)^TR_p^q &= \left[\left<\aq{p}_i^q, \aq{p}_j^q\right>\right] = \left[\delta_{ij}\right] \\
        &= I
    \end{align*}
    thus, $(R_p^q)^{-1} = (R_p^q)^T$
\end{proof}

\begin{theorem}
    The matrix representation of the rotation operator $\mathbb{R}_{ab}$ in two frames $\mathcal{F}_{\mathcal{V}}^a$ and $\mathcal{F}_{\mathcal{V}}^b$ is
    $$\left[\mathbb{R}_{ab}\right]^a = \left[\mathbb{R}_{ab}\right]^b = C_b^a$$
\end{theorem}

\begin{proof}
    \begin{align*}
        \left[\mathbb{R}_{ab}\right]^a = R_{ab}^a &= \left[\left<\mathbb{R}_{ab}\q{a}_i, \qs{a}_j\right>\right] \\
        &=\left[\left<\q{b}_i, \qs{a}_j\right>\right]\\
        &= C_b^a\\
    \end{align*}
    Using the "similarity transformation" of the $R_{ab}^a$:
    \begin{align*}
        R_{ab}^b &= C_a^bR_{ab}^aC_b^a \\
        &= C_a^b C_b^a C_b^a \\
        &= C_b^a
    \end{align*}
    And thus $R_{ab}^b = R_{ab}^a$

\end{proof}

\begin{theorem}
    The derivative of the rotation matrix $R_p^q$ is given by
    \begin{align*}
        \dot{R}_p^q &= S(\aq{w}_p^{qq})R_p^q \\
        &= R_p^qS(\aq{w}_p^{qp})
    \end{align*}
    \begin{proof}
        Since $R_p^q$ is a rotation matrix (orthonormal), we have
        \begin{align*}
            &(R_p^q)^{-1} = (R_p^q)^T \\
            \implies &R_p^q(R_p^q)^T = I
        \end{align*}
        Taking the derivative on both sides and applying the product rule gives
        \begin{align*}
            \dot{R_p^q}(R_p^q)^T + R_p^q(\dot{R_p^q})^T = 0
        \end{align*}
        We now define the matrix $S = \dot{R_p^q}(R_p^q)^T$ such that
        \begin{align*}
            S + S^T = 0
        \end{align*}
        This means S is some skew-symmetric matrix, with form
        \begin{align*}
            S(\aq{w}) = \begin{bmatrix}
                0 & -\omega_3 &\omega_2 \\ 
                \omega_3 &  0 & -\omega_1 \\
                -\omega_2 & \omega_1 & 0
            \end{bmatrix}
        \end{align*}
        From our definition, we have that $\dot{R_p^q} = S(\aq{\omega})R_p^q$. Finally, we want to find an interpretation of the vector $\aq{\omega}$. Writing this equation out by interpreting $R_p^q$ as an attitude matrix gives
        $$\begin{bmatrix} \dot{\aq{p}}_1^q & \dot{\aq{p}}_2^q & \dot{\aq{p}}_3^q\end{bmatrix} = S(\aq{\omega})\begin{bmatrix} {\aq{p}}_1^q & {\aq{p}}_2^q & {\aq{p}}_3^q\end{bmatrix}$$
        \begin{align*}
            \dot{\aq{p}}_i^q &= S(\aq{\omega})\aq{p}_i^q\\
            &= \aq{\omega} \times \aq{p}_i^q
        \end{align*}
        We therefore interpret $\aq{\omega}$ as the angular velocity of of the p-frame seen from the q-frame, and represented in the q-frame:
        $$\dot{R_p^q} = S(\aq{\omega}_p^{qq})R_p^q$$
        Since $S(\aq{\omega}_p^{qq})$ is a linear operator, we can apply express it using the similarity transform:
        \begin{align*}
            S(\aq{\omega}_p^{qq}) &= \left[\q{\omega}_p^q\times\right]^q\\
            &= R_p^q\left[\q{\omega}_p^q\times\right]^pR_q^p \\
            &= R_p^qS(\aq{\omega}_p^{qp})R_q^p
        \end{align*}
        Inserting this into the equation above gives
        \begin{align*}
            \dot{R_p^q} &= S(\aq{\omega}_p^{qq})R_p^q \\
            &= R_p^qS(\aq{\omega}_p^{qp})\cancel{R_q^pR_p^q} \\
            &= R_p^qS(\aq{\omega}_p^{qp})
        \end{align*}
    \end{proof}
\end{theorem}

\begin{theorem}
    The derivative of the DCM $C_p^q$ is given by
    \begin{align*}
        \dot{C}_p^q &= S(\aq{w}_p^{qq})C_p^q \\
        &= C_p^qS(\aq{w}_p^{qp})
    \end{align*}
    \begin{proof}
        $$C_p^q = \begin{bmatrix} {\aq{p}}_1^q & {\aq{p}}_2^q & {\aq{p}}_3^q\end{bmatrix}$$
        $$\dot{C_p^q} = \begin{bmatrix} {\dot{\aq{p}}}_1^q & \dot{\aq{p}}_2^q & \dot{\aq{p}}_3^q\end{bmatrix}$$
        Here we use the result from the proof above
        $$\dot{\aq{p}}_i^q = S(\aq{\omega}_p^{qq})\aq{p}_i$$
        $$\dot{C_p^q} = \begin{bmatrix} S(\aq{\omega}_p^{qq}){\aq{p}}_1^q & S(\aq{\omega}_p^{qq})\aq{p}_2^q & S(\aq{\omega}_p^{qq})\aq{p}_3^q\end{bmatrix}$$
        $$\dot{C_p^q} = S(\aq{\omega}_p^{qq})C_p^q$$
        Using the similarity transform we also get $\dot{C_p^q} = C_p^qS(\aq{\omega}_p^{qp})$
    \end{proof}
\end{theorem}

\begin{theorem}
    The derivative of a vector $\q{r}$ which is fixed in the rotating frame $\mathcal{F}_{\mathcal{V}}^p(t)$, seen from the (fixed) frame $\mathcal{F}_{\mathcal{V}}^q$ is
    $$\dot{\q{r}}^q = \q{\omega}_p^q\times\q{r}$$
\end{theorem}
\begin{proof}
    The vector $\q{r}$ can be expressed in the p-frame as
    $$\q{r} = \sum_{i=1}^3r_i^p\q{p}_i$$
    Where $r_i^p$ are constant, as the vector is fixed in the p-frame. Taking the time-derivative seen from the q-frame:
    $$\dot{\q{r}}^q = \sum_{i=3}^3r_i^p\dot{\q{p}}_i^q$$
    From previously, we have that $\dot{\q{p}}_i = \q{\omega}_p^{q} \times \q{p}_i$, where $\q{\omega}_p^q$ is the angular velocity of the the p-frame relative to the q-frame. Inserting this and using the fact that "$\q{\omega} \times$" is linear gives
    \begin{align*}
        \dot{\q{r}}^q &= \sum_{i=1}^3r_i^p(\q{\omega}_q^p\times\dot{\q{p}}_i^q) \\
        &=\q{\omega}_p^q \times\left( \sum_{i=1}^nr_i^p\q{p}_i^q \right)\\
        &= \q{\omega}_p^q \times \q{r}
    \end{align*}
\end{proof}

\begin{theorem}
    The derivative of a vector $\q{r}(t)$ which is time-varying in a rotating frame $\mathcal{F}_{\mathcal{V}}^p(t)$, seen from the (fixed) frame $\mathcal{F}_{\mathcal{V}}^q$ is
    $$\dot{\q{r}}^q = \q{\omega}_p^q\times\q{r}$$
    \begin{proof}
        We again coordinatize $\q{r}(t)$, now with time-varying coordinates $r_i^p(t)$:
        $$\q{r}(t) = \sum_{i=1}^nr_i^p(t)\q{p}_i(t)$$
        Taking the derivative seen from the q-frame and applying the product rule gives
        \begin{align*}
            \dot{\q{r}}^q &= \sum_{i=1}^{3}{\dot{r_i}^{pp}}\q{p}_i + \sum_{i=1}^{3}r_i^p\dot{\q{p}}_i^q \\
            \dot{\q{r}}^q &= \dot{\q{r}}^p + \q{\omega}_p^q\times\q{r}
        \end{align*}

    \end{proof}
\end{theorem}

\begin{theorem}
    Given two frames $\mathcal{F}_{\mathcal{V}}^q = \left\{ O_q; \q{q}_1, \q{q}_2, \q{q}_3\right\}$ and $\mathcal{F}_{\mathcal{V}}^q = \left\{ O_p; \q{p}_1, \q{p}_2, \q{p}_3\right\}$ and a point P. Let 
    \begin{align*}
        \q{r} &= P-O_q \\
        \q{\rho} &= P-O_p\\
        \q{r}_{qp} &= O_p-O_q \\
    \end{align*}
    Then, we have the following relations for the velocity and acceleration seen from the q- and p-systems:
        $$\q{r} = \q{r}_{qp} + \q{\rho} $$
        $$\dot{\q{r}}^q = \dot{\q{r}}_{qp}^q + \dot{\q{\rho}}^p + \q{\omega}_p^q \times \q{\rho} $$
        $$\ddot{\q{r}}^{qq} = \ddot{\q{r}}_{qp}^{qq} + \ddot{\q{\rho}}^{pp} + \dot{\q{\omega}}_p^{qq}\times\q{\rho} + \q{\omega}_p^q\times \left(\q{\omega}_p^q\times\q{\rho}\right)+2\q{\omega}_p^q\times \dot{\q{\rho}}^p$$
\end{theorem}
\begin{proof}
    The first relation of the position vectors can be computed directly
    \begin{align*}
        \q{r}_{qp} + \q{\rho} &= (O_p-O_q) + (P-O_p) \\
        \q{r}_{qp} + \q{\rho} &= \q{r}\\
    \end{align*}
    In this construction, the position vector $\q{\rho}$ and its derivatives can be considered as in the proof for theorem 9, so we have
    $$\dot{\q{\rho}}^q = \dot{\q{\rho}}^p + \q{\omega}_p^q\times\q{\rho}$$
    We take the q-frame derivative of both sides of the position vector equation, and apply this theorem recursively for $\q{\rho}$ and its derivatives, as well as the product rule for the cross-product derivatives
    \begin{align*}
        \q{r} &= \q{r}_{qp} + \q{\rho}  \\
        \dot{\q{r}}^q &= \dot{\q{r}}_{qp}^q + \dot{\q{\rho}}^q \\
        &= \underline{\dot{\q{r}}_{qp}^q + \dot{\q{\rho}}^p + \q{\omega}_p^q\times\q{\rho}}\\
    \end{align*}
    \begin{align*}
        \ddot{\q{r}}^{qq} &= \frac{{}^qd}{dt}\left[ \dot{\q{r}}_{qp}^q + \dot{\q{\rho}}^p + \q{\omega}_p^q\times\q{\rho} \right]\\
        &= \ddot{\q{r}}_{qp}^{qq} + \ddot{\q{\rho}}^{pq} +\left\{\dot{\q{\omega}}_p^{qq}\times\q{\rho}+\q{\omega}_p^q\times\dot{\q{\rho}}^q\right\}\\
        &= \ddot{\q{r}}_{qp}^{qq} + \left(\ddot{\q{\rho}}^{pp}+\q{\omega}_p^q\times\dot{\q{\rho}}^p\right) + \left\{ \dot{\q{\omega}}_p^{qq}\times\q{\rho} + \q{\omega}_p^q \times \left(\dot{\q{\rho}}^p + \q{\omega}_p^q\times\q{\rho}\right)\right\}\\
        &=\ddot{\q{r}}_{qp}^{qq} + \ddot{\q{\rho}}^{pp}+\q{\omega}_p^q\times\dot{\q{\rho}}^p + \dot{\q{\omega}}_p^{qq}\times\q{\rho} + \q{\omega}_p^q\times\dot{\q{\rho}}^p + \q{\omega}_p^q\times(\q{\omega}_p^q\times\q{\rho}) \\
        \ddot{\q{r}}^{qq}&= \underline{\ddot{\q{r}}_{qp}^{qq} + \ddot{\q{\rho}}^{pp} + \dot{\q{\omega}}_p^{qq}\times\q{\rho} + \q{\omega}_p^q\times \left(\q{\omega}_p^q\times\q{\rho}\right)+2\q{\omega}_p^q\times \dot{\q{\rho}}^p}
    \end{align*}
\end{proof}

\begin{theorem}
    Assume Newton's 3rd law holds for the force between particles, i.e $\q{f}_{ij} = -\q{f}_{ji}$.
    Then, the total external force $\q{F}$ is equal to the total mass $M$ times the acceleration of the 
    center of mass $\q{a}_c^{\mathbf{i}}$ seen from the inertial frame:
    $$\q{F} = M\frac{{}^{\mathbf{i}}d{}^{\mathbf{i}}d}{dt^2}\q{r}_c = M\q{a}_c^{\mathbf{i}}$$
    \begin{proof}
        The center of mass is defined as
        $$\q{r}_c = \frac{1}{M}\nsum{i}m_i\q{r}_i$$
        Where $m_i$ is the mass of particle $i$ with position vector $\q{r}_i$. The acceleration of the center of mass in the inertial frame $\q{a}_c^{\mathbf{i}}$ is found by differentiating twice
        \begin{align*}
            \q{a}_c^{\mathbf{i}} &= \ddot{\q{r}}_c^{\mathbf{ii}} \\
            &= \frac{1}{M}\nsum{i}m_i\ddot{\q{r}}_i^{\mathbf{ii}}\\
            \implies M\q{a}_c^{\mathbf{i}} &=\nsum{i}m_i{\q{a}}_i^{\mathbf{i}}
        \end{align*}
        Newton's law for a single particle states
        $$m_i\q{a}_i^{\mathbf{i}} = \q{F}_i$$
        Where $\q{F}_i$ is the total force on the particle, i.e is the sum of internal and external forces on the particle
        $$\q{F}_i = \q{f}_i + \nsum{j}\q{f}_{ji} = m_i\q{a}_i^{\mathbf{i}}$$
        Substituting this into the n-particle equation gives
        $$ M\q{a}_c^{\mathbf{i}} = \nsum{i}\left(\q{f}_i+\nsum{j}f_{ji}\right)$$
        Since $f_{ij} + f_{ji} = 0$, the contribution from internal forces is zero, and the sum reduces to
        \begin{align*}
            M\q{a}_c^{\mathbf{i}} &= \nsum{i}\q{f}_i \\
            M\q{a}_c^{\mathbf{i}} &= \q{F}
        \end{align*}
    \end{proof}
\end{theorem}

\begin{theorem}
    Let A be an arbitrary point in the interial frame $\mathbf{i}$ ($\q{r} = \q{r}_A+\q{\rho}_A$). Given a particle P acted on by a force F, the relation between the 
    torque and angular momentum about A is
    $$\q{n}_A = \dot{\q{h}}_A^{\mathbf{i}} + \rho_A\times(m\ddot{\q{r}}_A^{\mathbf{i}})$$
    Where the torque about A $\q{n}_A$ and the angular momentum about A in the inertial frame ${\q{h}}_A^{\mathbf{i}}$ are defined as:
    $$\q{n}_A = \q{\rho}_A\times\q{F}$$
    $${\q{h}}_A^{\mathbf{i}} = \q{\rho}_A\times(m\dot{\q{\rho}}_A^{\mathbf{i}})$$
    If $\rho_A\times(m\ddot{\q{r}}_A^{\mathbf{i}}) = 0$, i.e when either:
    \begin{enumerate}
        \item $\ddot{\q{r}}_A^{\mathbf{i}} = \q{0}$: A has constant velocity in the inertial frame (or not moving)
        \item $\q{\rho} \parallel \ddot{\q{r}}_A^{\mathbf{i}}$: A is accelerating towards/away from particle P
    \end{enumerate}
    Then the law of angular momentum can be written as
    $$\q{n}_A = \dot{\q{h}}_A^{\mathbf{i}}$$
    \begin{proof}
        Let $\q{r}$ be the position vector of the particle P, $\q{r}_A$ the vector for the arbitrary point A and $\q{\rho}$ from A to P. 
        $$\q{r} = \q{r}_A + \q{\rho}$$
        The torque on the particle P about a point A is defined as
        \begin{align*}
            \q{n}_A &= \q{\rho}_A\times\q{F} \\
            &=\q{\rho}_A\times (m\ddot{\q{r}}^{\mathbf{i}}) \\
            &=\q{\rho}_A\times \left(m\frac{{}^{\mathbf{i}}d^2}{dt^2}\left(\q{r}_A + \q{\rho}\right)\right) \\
            &= \q{\rho}_A \times (m\ddot{\q{r}}_A^{\mathbf{i}}) + \rho \times (m\ddot{\q{\rho}}^{\mathbf{i}})
        \end{align*}
        To complete the proof, we need to show that $\rho \times (m\ddot{\q{\rho}}^{\mathbf{i}})$ is equal to $\dot{\q{h}}_A^{\mathbf{i}}$.
        Using the definition of $\dot{\q{h}}_A^{\mathbf{i}}$ and computing its derivative:
        \begin{align*}
            {\q{h}}_A^{\mathbf{i}} &= \q{\rho}_A\times(m\dot{\q{\rho}}_A^{\mathbf{i}})\\
            \dot{\q{h}}_A^{\mathbf{i}} &= \dot{\q{\rho}}_A^{\mathbf{i}}\times(m\dot{\q{\rho}}_A^{\mathbf{i}}) + \q{\rho}_A\times(m\ddot{\q{\rho}}_A^{\mathbf{i}})
        \end{align*}
        The first term is $\q{0}$, since $\angle\dot{\q{\rho}}_A^{\mathbf{i}}\dot{\q{\rho}}_A^{\mathbf{i}} = 0$. So we have
        $$\dot{\q{h}}_A^{\mathbf{i}} = \q{\rho}_A\times(m\ddot{\q{\rho}}_A^{\mathbf{i}})$$
        Finally, inserting this into the equation for $\q{n}_A$ gives
        $$\q{n}_A = \dot{\q{h}}_A^{\mathbf{i}} +  \rho \times (m\ddot{\q{\rho}}^{\mathbf{i}})$$
    \end{proof}
\end{theorem}

\begin{theorem}
    Given a system of n particles where particle $i$ is acted on by an external force $\q{F}_i$ and the force $\q{f}_{ij}$ from particle j for which
    Newton's 3rd law is assumed to hold. In addition, these forces $\q{f}_{ji} = -\q{f}_{ij}$ are central forces (point along $\q{r}_i-\q{r}_j$).
    
    Then, for an arbitrary point A in the inertial frame $\mathbf{i}$, we have the following relations
    $$\q{n}_A = \dot{\q{h}}_A^{\mathbf{i}} + \nsum{i}m_i\q{\rho}_{Ai}\times\ddot{\q{r}}_A^{\mathbf{i}}$$
    Where the total torque and total angular momentum about A are:
    $$\q{n}_A = \nsum{i}\q{\rho}_{Ai}\times{\q{F}}_i$$
    $$\q{h}_A = \nsum{i}m_i\q{\rho}_{Ai}\times\dot{\q{\rho}}_{Ai}^{\mathbf{i}}$$
    If $\nsum{i}m_i\q{\rho}_{Ai}\times\ddot{\q{r}}_A^{\mathbf{i}} = 0$, i.e when either:
    \begin{enumerate}
        \item $\nsum{i}m_i\q{\rho}_{Ai} = \q{0}$: A is placed in the center of mass
        \item $\ddot{\q{r}}_A^{\mathbf{i}} = \q{0}$: A has constant velocity
        \item $\q{\rho}_{Ai} \parallel \ddot{\q{r}}_A^{\mathbf{i}}$: A is accelerating towards the center of mass
    \end{enumerate}
    Then, the law of angular momentum for the n-particle system can be written as
    $$\q{n}_A = \dot{\q{h}}_A^{\mathbf{i}}$$
    \begin{proof}
        We have the following relation for the position vectors:
        $$\q{r}_i = \q{r}_A+\q{\rho}_i$$

        \subsubsection*{Total torque}
        The torque on a single particle $i$ about $A$ is defined as:
        \begin{align*}
            \q{n}_{Ai} = \q{\rho}_{Ai}\times\left(F_i + \nsum{i}\q{f}_{ij}\right)
        \end{align*}
        The total torque $\q{n}_A$ is therefore
        $$\q{n}_A = \nsum{i}\left(\q{\rho}_{Ai}\times \q{F}_i + \nsum{j}\q{\rho}_{Ai}\times\q{f}_{ij}\right)$$
        We want to show that all torque contributions from internal forces cancel out.
        For any term (i, j) in the sum, there is an associated term (j, i). We want to show that the internal torques from these terms cancel out:
        \begin{align*}
            (i, j):&{}\\
            &{}\q{\rho}_{Ai}\times\q{f}_{ij} = (\q{r}_i-\q{r}_A)\times\q{f}_{ij}\\
            (j, i):&{}\\
            &{}\q{\rho}_{Ai}\times\q{f}_{ij} = (\q{r}_j-\q{r}_A)\times(-\q{f}_{ij})\\
            (i, j) + (j, i):&{}\\
            &{}(\q{r}_i-\q{r}_A)\times\q{f}_{ij} - (\q{r}_j-\q{r}_A)\times\q{f}_{ij} \\
            &=(\q{r}_i-\q{r}_j)\times{\q{f}_{ij}} \\
            &=\q{0}
        \end{align*}
        Here, we used the fact that $\q{f}_{ij}=-\q{f}_{ji}$ and $\q{f}_{ij} \parallel (\q{r}_i-\q{r}_j)$.
        This reduces the total torque to
        $$\q{n}_A = \nsum{i}\q{\rho}_{Ai}\times\q{F}_i$$
        \subsubsection*{Total angular momentum}
        The angular momentum on a single particle i is defined as:
        $$\q{h}_{Ai}^{\mathbf{i}} = m_i\q{\rho}_{Ai}\times\dot{\q{\rho}}_{Ai}^{\mathbf{i}}$$
        Thus the total angular momentum is simply:
        $$\q{h}_{Ai}^{\mathbf{i}} =\nsum{i} m_i\q{\rho}_{Ai}\times\dot{\q{\rho}}_{Ai}^{\mathbf{i}}$$
        \subsubsection*{Relating total torque and angular momentum}
        We compute the derivative of the total angular momentum seen from the inertial frame:
        \begin{align*}
            \dot{\q{h}}_{Ai}^{\mathbf{i}} &= \nsum{i}m_i\left(\dot{\q{\rho}}_{Ai}^{\mathbf{i}}\times\dot{\q{\rho}}_{Ai}^{\mathbf{i}} +{\q{\rho}}_{Ai}^{\mathbf{i}}\times \ddot{\q{\rho}}_{Ai}^{\mathbf{i}}\right)
            \\ &= \nsum{i}\q{\rho}_{Ai}^{\mathbf{i}} \times (m_i\ddot{\q{\rho}}_{Ai}^{\mathbf{i}})
        \end{align*}
        We then consider the total torque $\q{n}_A$
        \begin{align*}
            \q{n}_A &= \nsum{i}\q{\rho}_{Ai}\times\q{F}_i\\
            &=\nsum{i}\q{\rho}_{Ai}\times(m_i\ddot{\q{r}}_i^{\mathbf{i}}) \\
        \end{align*}
        Since the position vector of the particle i is $\q{r}_i = \q{r}_A + \q{\rho}_{Ai}$, this becomes
        \begin{align*}
            \q{n}_A = \nsum{i} \q{\rho}_{Ai} \times (m\ddot{\q{r}}_A^{\mathbf{i}})
            + \underbrace{\q{\rho}_{Ai} \times (m\ddot{\q{\rho}}_{Ai}^{\mathbf{i}})}_{\dot{\q{h}}_{Ai}^{\mathbf{i}}}
        \end{align*}
        Thus
        \begin{align*}
            \q{n}_A = \dot{\q{h}}_{Ai}^{\mathbf{i}} + \nsum{i} \q{\rho}_{Ai} \times (m\ddot{\q{r}}_A^{\mathbf{i}})
        \end{align*}
    \end{proof}
\end{theorem}
\end{document}